\documentclass[a4paper,12pt]{jreport}
\usepackage{ascmac}
\usepackage{amsmath}
\usepackage{amssymb}
\usepackage{amsthm}
\usepackage{blkarray}
\usepackage{latexsym}
\usepackage[dvipdfmx]{graphicx}
\usepackage{tikz}
\usepackage{bm}
\usetikzlibrary{arrows.meta}
\usetikzlibrary{intersections,calc}
\newcounter{nombre}

\makeatletter
\def\Left#1#2\Right{\begingroup%
   \def\ts@r{\nulldelimiterspace=0pt \mathsurround=0pt}%
   \let\@hat=#1%
   \def\sht@im{#2}%
   \def\@t{{\mathchoice{\def\@fen{\displaystyle}\k@fel}%
          {\def\@fen{\textstyle}\k@fel}%
          {\def\@fen{\scriptstyle}\k@fel}%
          {\def\@fen{\scriptscriptstyle}\k@fel}}}%
   \def\g@rin{\ts@r\left\@hat\vphantom{\sht@im}\right.}%
   \def\k@fel{\setbox0=\hbox{$\@fen\g@rin$}\hbox{%
      $\@fen \kern.3875\wd0 \copy0 \kern-.3875\wd0%
      \llap{\copy0}\kern.3875\wd0$}}%
      \def\pt@h{\mathopen\@t}\pt@h\sht@im%
      \Right}%
\def\Right#1{\let\@hat=#1%
   \def\st@m{\mathclose\@t}%
   \st@m\endgroup}
\makeatother

\def \tp {{\rm tp}}
\def \stp{{\rm stp}}
\def \acl {{\rm acl}}
\def \dcl {{\rm dcl}}
\def \cl {{\rm cl}}
\def \ccl {{\rm ccl}}

\def \dom {{\rm dom}}
\def \ran {{\rm ran}}
\def \ind {\mathop{\smile \hskip -0.9em ^| \ }}
\def \dep {\mathop{\not \!\! \ind}}
\def \non-ort {{\not\!\!\bot}}
\def \domeq {\sqsubseteq\!\!\!\!\sqsupseteq}

\def \Th {{\rm Th}}
\def \RM {{\rm RM}}
\def \Z  {{\mathbb Z}}
\def \N {{\mathbb N}}
\def \Q {{\mathbb Q}}
\def \R {{\mathbb R}}
\def \C {{\mathbb C}}
\def \M {{\cal M}}
\def \U {{\rm U}}
\def \S#1 {{\rm S}#1 }
\def \lh{{\rm lh}}

\def \ob {{\rm Ob}}
\def \id {{\rm id}}

\def \imp {{\Rightarrow}}
\def \notequiv {{\hskip 0.1em \not\hskip -0.1em \leftrightarrow}}
\def \tie{ \,{\widehat { \ }}\, }

\def\proclaim #1. #2\par{\medbreak
  \noindent{\bf#1 \enspace}{\sl#2}\par
  \ifdim\lastskip<\medskipamount \removelastskip\penalty55\medskip\fi}
\renewcommand\thefootnote{*\arabic{footnote}}
\newtheorem{theo}{Theorem}[section]
\newtheorem{coro}[theo]{Corollary}
\newtheorem{prop}[theo]{Proposition}
\newtheorem{rem}[theo]{Remark}
\newtheorem{exam}[theo]{example}
\newtheorem{exer}[theo]{Exercise}
\newtheorem{lem}[theo]{Lemma}
\newtheorem{fact}{fact}
\newtheorem{claim}{Claim}[theo]
\newtheorem{problem}{問}[chapter]
\renewcommand{\theclaim}{\Alph{claim}}
\renewcommand{\thefact}{}
\newtheorem{shelah}{Shelah's conjecture}
\theoremstyle{definition}
\newtheorem{defi}{定義}
\newcommand{\prob}[1][]{\refstepcounter{nombre}#1(\thenombre)}
\newcommand{\ang}[1]{#1^\circ}
\renewcommand{\theshelah}{}

\title{行列について}
\author{中川路 玄哉}

\begin{document}
   \maketitle
   \tableofcontents
   \chapter{行列とは?}
   \section{座標平面の復習}\label{coordinate}
   Sさん「先生、行列ってなんですか?」\\

   T先生「行列は線型空間から線形空間への写像だね.」\\
   
   Sさん「は?…ちょっとよくわかりません. もうちょっとわかる言葉で教えてください.」\\

   T先生「うーん, ちなみにSさんはベクトルは知っているのかい?」\\

   Sさん「ベクトル…言葉ぐらいは聞いたことあります.」\\

   T先生「そうか, じゃあベクトルはやめて座標平面を例にあげよう.」\\

   Sさん「座標平面っていうと, あの$x$座標$y$座標っていうやつですか?」\\

   T先生「そうだね. 少し復習しようか. まずは座標$(2, 3)$を$x$軸方向に$1$, $y$軸方向に$2$平行移動した点は?」\\

   Sさん「$(3, 5)$です!」\\

   T先生「正解!じゃあ, 座標$(1, 2)$を原点から見た方向はそのままに, 原点からの距離を2倍した位置の座標は?」\\

   Sさん「えーと, 両方を2倍して, $(2, 4)$でしょうか.」\\

   T先生「よく出来てるよ. じゃあ次はちょっと難しく, 座標$(2,2)$を原点からの距離はそのままで, 時計と反対回りに$\ang{60}$回転させた位置の座標は?」\\

   Sさん「え!?うーん, わかりません...」\\

   T先生「ちょっと難しかったね. わからなくても全然大丈夫だよ.」\\

   少し座標平面の復習をしましょう.
   まず最初に抑えておいて欲しいのが, 平行移動です.
   T先生とSさんの例のように"$x$軸方向に$1$, $y$軸方向に$2$移動させる"という操作は座標$(2, 3)$を座標$(3, 5)$に平行移動させます.
   
   これは, 具体的な数字じゃなくても同じことが言えます.
   上記操作は座標$(x, y)$を$(x+1, y+2)$に平行移動させるわけです.
   
   具体的な数字である必要がないということは, 関数でも同じことが言えることを意味します.
   $y=x^2$という関数について考えて見ましょう.
   同様に先ほどの操作で平行移動することをグラフで表現すると以下のようになります.

   \begin{center}
      \begin{tikzpicture}
         \draw[thick, ->] (-3,0) -- (3,0) node [below] {$x$};%x軸
         \draw[thick, ->] (0,-1) -- (0,4) node [left] {$y$};%y軸
         \node at (0,0) [anchor=north east] {O};
         \draw[domain=-2:2] plot (\x, \x * \x);
         \draw[domain=-1.4142:1.4142] plot (\x + 1, \x * \x + 2);
         \draw[->] (0,0) -- (1,0) node [below][pos=0.5] {$+1$};
         \draw[->] (1,0) -- (1,2) node [right][pos=0.5] {$+2$};
      \end{tikzpicture}
   \end{center}

   平行移動した後の関数がどういった式で表現できるかと言うと,
   $x$に$x-1$, $y$に$y-2$を代入すれば元どおり等式が成り立つはずなので,
   $$y-2=(x-1)^2$$
   となり, 式を綺麗に整えると$y=(x-1)^2+2$となります.

   同様に,
   $y=x^2$のグラフの全ての点を方向は保ったまま原点からの距離を2倍にしたい場合を考えてみましょう。
   座標$(x,y)$に対する原点からの距離を2倍にした点の座標は$(2x,2y)$です.
   先ほどと同じで等式を成り立たせるためには$x$に$\frac{1}{2}x$を,
   $y$に$\frac{1}{2}y$を代入すればいいので,
   $$\frac{1}{2}y=\frac{1}{4}x^2$$
   となり, 式を綺麗に整えると$y=\frac{1}{2}x^2$となります.
   
   次は座標$(2,2)$を原点からの距離はそのままで,
   時計と反対回りに$\ang{60}$回転させた位置の座標について考えてみましょう.
   
   \begin{center}
      \begin{tikzpicture}
         \draw[thick, ->] (-3,0) -- (3,0) node [below] {$x$};%x軸
         \draw[thick, ->] (0,-1) -- (0,4) node [left] {$y$};%y軸
         \coordinate(O) at (0,0) node at (0,0) [anchor=north east] {O};
         \coordinate(A) at (2,2) node [right] at (A) {$(2,2)$};
         \coordinate(B) at (-0.73205,2.73205) node [right] at (B) {$(?,?)$};
         \fill (A) circle (2pt);
         \fill (B) circle (2pt);
         \draw (B)--(O)--(A);
         \draw (0.3, 0.3) arc [start angle=45, end angle=128, radius=0.3];
         \node at (0.2,0.6) {$\ang{60}$};
      \end{tikzpicture}
   \end{center}

   ここで思い出して欲しいのが三角比です.
   正弦は斜辺分の高さで,
   余弦は斜辺分の底辺でした.
   斜辺が1の場合は正弦がそのまま高さになり,
   余弦はそのまま底辺になります.

   正弦と余弦は座標平面ではそのまま半径1の円上の座標になります.
   どういうことかと言うと,
   半径1の円上の点$P$の座標は,
   点$P$と原点を結ぶ直線と$x$軸のなす角を$\theta$としたとき,
   $(\cos{\theta}, \sin{\theta})$となります.

   この性質を使っていきましょう.
   まず半径1の円上の性質だったため,
   半径1の円に持ち込む必要があります.
   どんな点も各座標を原点からの距離で割ることで,
   原点からの距離が1になり,
   半径1の円上の点となります.
   座標$(2,2)$の場合,
   原点からの距離は$2\sqrt{2}$なので,
   $(\frac{1}{\sqrt{2}},\frac{1}{\sqrt{2}})$が方向は同じでかつ半径1の円上の点となります.
   ここからわかることは$(2,2)$と原点を結ぶ直線と$x$軸のなす角を$\theta$としたとき,
   $$\cos{\theta}=\frac{1}{\sqrt{2}}, \sin\theta=\frac{1}{\sqrt{2}}$$
   ということです.

   次に求めたい座標のある方向を$x$軸からの角度で考えて見ると,
   先ほどの$\theta$を使って,
   $\theta+\ang{60}$であることがわかります.
   つまり,
   $(\cos\theta+\ang{60},\sin\theta+\ang{60})$を求めれば,
   あとは長さを調節するだけとなります.
   加法定理を使うと,
   $$\left(\frac{1-\sqrt{3}}{2\sqrt{2}},\frac{1+\sqrt{3}}{2\sqrt{2}}\right)$$
   となります.
   
   最後に原点からの距離を$(2,2)$と同じにする必要があるので,
   $2\sqrt{2}$をかけることによって,
   求める座標は,
   $$\left(1-\sqrt{3},1+\sqrt{3}\right)$$
   であることがわかります.

   \begin{problem}
      座標$(-1,4)$を$x$軸方向に$-1$, $y$軸方向に$2$平行移動した点の座標を答えよ.
   \end{problem}
   \begin{problem}
      関数$y=x^3+x$を$x$軸方向に$2$倍, $y$軸方向に$3$倍引き伸ばした関数を答えよ.
   \end{problem}
   \begin{problem}
      座標$(x,2^x)$を原点からの距離はそのままに時計回りに$\ang{30}$回転させた点の座標を答えよ.
   \end{problem}
   \begin{problem}
      任意の点$(x,y)$について, $(x,y)=a(1,0)+b(1,2)$となる実数の組$(a,b)$が存在することを証明せよ.
   \end{problem}

   \newpage

   \section{行列の正体}\label{sense}

   Sさん「座標平面を復習しましたけど,
   これが行列と何の関係があるんですか?」\\

   T先生「実は行列っていうのは,
   さっきからやっていた座標変換そのものなんだ.」\\

   Sさん「うーん,
   すみません,
   ピンときていません...」\\

   T先生「さっきまでの座標変換はそれぞれ違う手法があって,
   あまり共通した処理という気がしなかっただろう?」\\

   Sさん「そうですね,
   平行移動, 引き伸ばし, 回転とそれぞれにやり方があるイメージです.」\\

   T先生「実はその全ての処理は行列でしかなく,
   行列の積で表現できてしまうんだ.」\\

   Sさん「なるほど!それぞれの処理に対してどう対応するかではなく,
   座標変換といえばこうするという1つの対応方法で対応できてしまうということですね.」\\

   T先生「そういうことだね.
   こういう風にバラバラなものの共通した性質を抜き出してまとめることを一般化と言うよ.
   行列はあらゆる座標変換の一般化になっているわけだ.」\\

   この節では行列の気持ちを理解しましょう.
   T先生とSさんの話では,
   行列は座標変換の一般化だと言う話でしたが,
   ついでなので, 座標側も一般化しましょう.
   座標を一般化したものをベクトルと言います.
   扱うのが実数の範囲であれば,
   座標とベクトルは同じものなので,
   今回は同じものと考えて大丈夫です.
   ただ,
   $n$次元空間の点の座標と言われてもピンときませんが,
   $n$次元のベクトルと言われると親しみやすかったりするので,
   そういった理由で今後はベクトルと呼びます.

   $n$次元の実数ベクトルと言われると$n$個の実数値を並べたものを意味し,
   例えば$n$次元実数ベクトル$\bm{a}$は以下のように表記されたりもします.
   $$\bm{a}=\left(a_0,a_1,a_2,\dots,a_n\right)\ (a_0,\dots,a_n\rm{は実数})$$
   ベクトルについては以下の2つのことを覚えておけば十分です.
   \begin{enumerate}
      \item $\bm{a}=(a_0,\dots,a_n),\bm{b}=(b_0,\dots,b_n)$を$n$次元の実数ベクトルとする.\\
         この時, ベクトルの和$\bm{a}+\bm{b}$は以下のように定義される.
         $$\bm{a}+\bm{b}=(a_0+b_0,\dots,a_n+b_n)$$
      \item $\bm{a}=(a_0,\dots,a_n)$を$n$次元の実数ベクトルとする.
         この時, ベクトルの定数倍$k\bm{a}\ (k\rm{は実数})$は以下のように定義される.
         $$k\bm{a}=(ka_0,\dots,ka_n)$$
   \end{enumerate}

   見てわかる通り,
   2次元の座標平面でやっていたことと何も変わりません.
   これに対して, 行列を用意して, 行列の積を取ることによってベクトル変換を表現することができます.
   例えば, 節\ref{coordinate}で扱った反時計回りに$\theta$回転させる変換を表現した行列は,
   $$
   \begin{pmatrix}
      \cos\theta & -\sin\theta \\
      \sin\theta & \cos\theta
   \end{pmatrix}
   $$
   となります.
   実際にベクトル$(x,y)$の変換先を知りたい場合は,
   行列の積(後で定義する)を使って導くことができます.
   $$
   \begin{pmatrix}
      \cos\theta &-\sin\theta \\
      \sin\theta & \cos\theta
   \end{pmatrix}
   \begin{pmatrix}
      x \\
      y
   \end{pmatrix}
   =
   \begin{pmatrix}
      x\cos\theta-y\sin\theta \\
      x\sin\theta+y\cos\theta
   \end{pmatrix}
   $$
   上記計算によって, 変換先のベクトルが$(x\cos\theta-y\sin\theta,x\sin\theta+y\cos\theta)$であることがわかります.
   このように, 複数行と複数列で値を表記したものを行列と言います.
   
   もちろん,
   ベクトル同士の和や, ベクトルの定数倍によって,
   ベクトルは別のベクトルへ変換することができるが,
   このような操作は計算前の次元数と計算結果の次元数が同じものに限っています.
   行列による変換は次元が異なるベクトルへの変換も可能です.
   例えば以下のような行列では,
   2次元のベクトルを3次元のベクトルへ変換しています.
   $$
   \begin{pmatrix}
      1 & 0\\
      0 & 1\\
      1 & 1
   \end{pmatrix}
   \begin{pmatrix}
      x\\
      y
   \end{pmatrix}
   =
   \begin{pmatrix}
      x\\
      y\\
      x+y
   \end{pmatrix}
   $$

   まとめると, この節では以下の2つのことを理解してもらえれば十分だと思います.
   \begin{enumerate}
      \item ベクトルには和と定数倍が定義されており, それらは同じ次元内で完結している話であること.
      \item 行列とは, ベクトルからベクトルへの変換であること.
   \end{enumerate}

   \begin{problem}
      $2(1,5)-3(2,1)$を求めよ.
   \end{problem}
   \begin{problem}
      行列
      $
      \begin{pmatrix}
         1 & 0\\
         0 & 1\\
         1 & 1
      \end{pmatrix}
      $
      によるベクトル$(x,y)$の演算結果は以下の通りである.
      $$
      \left(
      \begin{pmatrix}
         1 & 0\\
         0 & 1\\
         1 & 1
      \end{pmatrix}
      \begin{pmatrix}
         x\\
         y
      \end{pmatrix}
      \right)^t
      =
      (x,y,x+y)\\
      $$
      (右上についている$t$はちゃんとした定義は後ほど行うが, ベクトルを縦向きに書くか横向きに書くかを合わせるためのものである)\\
      この時, 以下を求めよ.
      $$
      (2,3,1)+
      \left(
      \begin{pmatrix}
         1 & 0\\
         0 & 1\\
         1 & 1
      \end{pmatrix}
      \begin{pmatrix}
         1\\
         2
      \end{pmatrix}
      \right)^t
      $$
   \end{problem}
   \begin{problem}
      $\bm{a},\bm{b},\bm{c}$を$n$次元実数ベクトルとし, $k$を実数とする.
      この時, 以下の$3$つの等式が成り立つことを示せ.
      \begin{eqnarray*}
         \bm{a}+\bm{b}=\bm{b}+\bm{a}\\
         (\bm{a}+\bm{b})+\bm{c}=\bm{a}+(\bm{b}+\bm{c})\\
         k(\bm{a}+\bm{b})=k\bm{a}+k\bm{b}
      \end{eqnarray*}
   \end{problem}

   \chapter{行列の計算}
   \section{積と転置}
   T先生「さて, 行列を扱う為に必要な概念を順番に用意していこうか.」\\

   Sさん「はい, よろしくお願いします.」\\

   T先生「まずは行列そのものについてだけど, 今回扱うのは実数行列だけなので, 行列と言われたら暗黙の了解で実数行列のことを言うことにしよう.」\\

   Sさん「わかりました. たしか, 行列は縦と横にいくつか数字を並べたものでしたよね?」\\

   T先生「そうだね. 行が$n$行, 列が$m$列の行列を$n\times m$行列と呼ぶよ.」\\

   Sさん「行列の積によってベクトルを変換できるとの話でしたが, ベクトルも$1$行もしくは$1$列の行列だと思うことができると言うことでしょうか.」\\

   T先生「まさにその通りだよ. 計算するときはベクトルも一つの行列として扱うんだ.」\\

   行列の積について考えてみましょう.
   行列の積は$m\times n$行列と$n\times l$行列といった具合に,
   前の行列の列数と後ろの行列の行数が一致する場合にのみ定義することができます.
   実際の定義は以下の通りです.
   \begin{defi}[行列の積]
      以下の通り2つの行列$A, B$を用意する.
      $$
      A=
      \begin{pmatrix}
         a_{00} &a_{01} &a_{02}& \cdots &a_{0n}\\
         a_{10} &a_{11} &\cdots& \cdots &a_{1n}\\
         \vdots &\vdots &\ddots& \ddots &\vdots\\
         a_{m0} &a_{m1} &\cdots &\cdots &a_{mn}
      \end{pmatrix}
      $$
      $$
      B=
      \begin{pmatrix}
         b_{00} &b_{01} &b_{02}& \cdots &b_{0l}\\
         b_{10} &b_{11} &\cdots& \cdots &b_{1l}\\
         \vdots &\vdots &\ddots& \ddots &\vdots\\
         b_{n0} &b_{n1} &\cdots &\cdots &b_{nl}
      \end{pmatrix}
      $$
      この時, 積を取った行列$AB$の$i$行$j$列目の要素は以下の通り定義することができる.
      $$
      a_{i0}b_{0j}+a_{i1}b_{1j}+\cdots+a_{in}b_{nj}
      $$
   \end{defi}

   実際の計算について, 上記定義を見るだけで理解するのは実質不可能なので,
   具体的に計算をしながら慣れていくことをおすすめします.
   ここでは, 行列の積が満たす性質を確認していきましょう.

   まず, $m\times n$行列と$n\times l$行列の積の結果は$m\times l$行列となります.
   ここからわかるように, 行列の積は順番が逆だと定義できない可能性もあります.
   例え定義できたとしても結果が異なる場合が多いです.

   加えて, 行数と列数が同じ行列のことを正方行列と言います.
   $n$行$n$列の行列を$n$次正方行列と呼んだりもします.
   $n$次正方行列同士の掛け算は順番に寄らず必ず定義できますが, 順番を入れ替えてしまうと結果が変わってしまう場合がほとんどです.
   また, 以下のような正方行列について考えてみましょう.
   $$
   \begin{pmatrix}
      1 &0 &\cdots & 0\\
      0 & 1&\cdots & 0\\
      \vdots & \ddots& \ddots &\vdots\\
      0 & 0 & \cdots & 1
   \end{pmatrix}
   $$
   見ての通り, 行数と列数が一致する場所は$1$, そうでない場所は$0$であるような行列です.
   この行列は, 左からかけても右からかけても元の行列を変化させることはありません.
   例えば, $A$を$n$次正方行列とし, 上記行列を$E$とすると,
   $$
   AE=EA=A
   $$
   が成り立ちます.
   この行列$E$を単位行列と言います.

   行列の積になれる為に節\ref{sense}で扱った時計回りに$\theta$回転させる操作を実際に計算してみましょう.
   $\theta$回転させる操作を表した行列は
   $$
   \begin{pmatrix}
      \cos\theta &-\sin\theta\\
      \sin\theta &\cos\theta
   \end{pmatrix}
   $$
   だったので, この行列を
   $
   \begin{pmatrix}
      x\\
      y
   \end{pmatrix}
   $
   にかけてみましょう.
   まず, この掛け算は$2\times 2$行列と$2\times 1$行列の積なので, 結果の行列は$2\times 1$行列になるはずです.
   では, 結果の行列の1行1列目がどうなるかを考えてみると,
   $
   x\cos\theta-y\sin\theta
   $
   となります.
   同様に, 2行1列目がどうなるか考えてみると
   $
   x\sin\theta+y\cos\theta
   $
   となります.
   したがって, 求める行列は,
   $$
   \begin{pmatrix}
      x\cos\theta-y\sin\theta\\
      x\sin\theta+y\cos\theta
   \end{pmatrix}
   $$
   となります.

   もう一つ概念を学んでおきましょう.
   行列の転置と言うものがあります.
   \begin{defi}[行列の転置]
      $A,P$を行列とする.
      $A$の$i$行$j$列目を$a_{ij}$, $P$の$i$行$j$列目を$p_{ij}$とした時,
      任意の$i,j$について,
      $$
      p_{ij}=a{ji}
      $$
      が成り立つ時, $P$を$A$の転置行列と呼び, $A^t$と書く.
   \end{defi}
   要約すると, 斜めのラインを軸として反転させた行列のことを転置行列と呼びます.
   行列$A$が$m\times n$行列だったとすると, $A^t$は$m\times n$行列になります.

   \begin{problem}
      以下の行列の積を求めよ.
      $$
      \begin{pmatrix}
         1 & 2 & 3\\
         4 & 5 & 6
      \end{pmatrix}
      \begin{pmatrix}
         2 & 6\\
         1 & 2\\
         3 & 3
      \end{pmatrix}
      $$
   \end{problem}
   \begin{problem}
      $\theta_0, \theta_1$を角度とする.
      $\theta_0$回転させた点からさらに$\theta_1$回転させた点と$\theta_0+\theta_1$回転させた点が一致することを確認せよ.
   \end{problem}

   \newpage

   \section{和と行列式}

   \chapter{行列の固有値}
   \section{固有値と固有ベクトル}

\end{document}